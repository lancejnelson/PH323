\documentclass{article}
\usepackage{amsmath}
%\usepackage{pstricks}
\usepackage{epic}
\begin{document}
\pagestyle{empty}



We seek the phonon dispersion curves for a crystal of Cu (which forms
an fcc lattice) in the $[110]$ direction in k-space.  A general vector
that points in that direction is given by:
\begin{equation}
  \mathbf{k} = {k\over \sqrt{2}} (\hat{i} + \hat{j})
\end{equation}

The expression for the force on the atom located at the origin due to
it's motion and the motion of on of it's neighbors is given by:
\begin{equation}
  \mathbf{F} = - \alpha \left[ \mathbf{\hat{R}}\cdot \mathbf{u}(000) -
  \mathbf{\hat{R}}\cdot \mathbf{u}(\mathbf{R})\right]\mathbf{\hat{R}}
\end{equation}

Each cu atom has 12 nearest neighbors.  They are located at:

\begin{eqnarray}
  a({1\over 2}{1\over 2}0) & a({1\over 2}0{1\over 2}) & a(0{1\over 2}{1\over 2})\\ 
  a(-{1\over 2}{1\over 2}0) & a(-{1\over 2}0{1\over 2}) & a(0-{1\over 2}{1\over 2})\\ 
  a({1\over 2}-{1\over 2}0) & a({1\over 2}0-{1\over 2}) & a(0{1\over 2}-{1\over 2})\\ 
  a(-{1\over 2}-{1\over 2}0) & a(-{1\over 2}0-{1\over 2}) & a(0-{1\over 2}-{1\over 2})\\ 
\end{eqnarray}


Let's attack them one at a time.


For $\mathbf{r}_n = a({1\over
  2}{1\over 2}0)$:

\begin{equation}
  \mathbf{\hat{R}} = {1\over \sqrt{2}} (\hat{i} + \hat{j})
  \end{equation}

and  assuming
\begin{align}
  \mathbf{u}_n &= \mathbf{A}e^{i \mathbf{k} \cdot \mathbf{r}_n - i \omega t}\\
  &= \mathbf{A}e^{i ({k a\over 2 \sqrt{2}} +{k a\over 2 \sqrt{2}} ) - i \omega t}\\
  &= \mathbf{A}e^{i ({k a\over \sqrt{2}} ) - i \omega t}\\
  &= \mathbf{u}(000) e^{i ({k a\over \sqrt{2}} )}\\
  \end{align}

\begin{align}
  \mathbf{F} &= - \alpha \left[ \mathbf{\hat{R}}\cdot \mathbf{u}(000) -
  \mathbf{\hat{R}}\cdot
  \mathbf{u}(\mathbf{R})\right]\mathbf{\hat{R}}\\
  &=- \alpha \left[ {1\over \sqrt{2}} (\hat{i} + \hat{j}) \cdot
    \mathbf{u}(000) - {1\over \sqrt{2}} (\hat{i} + \hat{j}) \cdot
    \mathbf{u}({1\over 2}{1\over 2}0)\right]{1\over \sqrt{2}}(\hat{i}
    + \hat{j})\\
  &=- {\alpha\over 2} \left[ (\hat{i} + \hat{j}) \cdot
    \mathbf{u}(000) - (\hat{i} + \hat{j}) \cdot
    \mathbf{u}({1\over 2}{1\over 2}0)\right](\hat{i}
    + \hat{j})\\
  &=- {\alpha\over 2} \left[u_x(000) + u_y(000) - u_x(000) e^{i ({k
    a\over \sqrt{2}} )} - u_y(000)e^{i ({k a\over \sqrt{2}} )} \right](\hat{i}
    + \hat{j}) \\
  &=- {\alpha\over 2} \left[ u_x(000)\left( 1 - e^{i ({k
    a\over \sqrt{2}} )} \right) + u_y(000)\left(1- e^{i ({k a\over \sqrt{2}} )}\right) \right](\hat{i}
    + \hat{j}) \label{eq:F1}\\
\end{align}



For $\mathbf{r}_n = a(-{1\over
  2}{1\over 2}0)$:

\begin{equation}
  \mathbf{\hat{R}} = {1\over \sqrt{2}} (-\hat{i} + \hat{j})
  \end{equation}

and  assuming
\begin{align}
  \mathbf{u}(-{1\over  2}{1\over 2}0) &= \mathbf{A}e^{i \mathbf{k} \cdot \mathbf{r}_n - i \omega t}\\
  &= \mathbf{A}e^{i ({-k a\over 2 \sqrt{2}} +{k a\over 2 \sqrt{2}} ) - i \omega t}\\
  &= \mathbf{A}e^{i 0 - i \omega t}\\
  &= \mathbf{u}(000) \\
  \end{align}

\begin{align}
  \mathbf{F} &= - \alpha \left[ \mathbf{\hat{R}}\cdot \mathbf{u}(000) -
  \mathbf{\hat{R}}\cdot
  \mathbf{u}(\mathbf{R})\right]\mathbf{\hat{R}}\\
  &=- \alpha \left[ {1\over \sqrt{2}} (-\hat{i} + \hat{j}) \cdot
    \mathbf{u}(000) - {1\over \sqrt{2}} (-\hat{i} + \hat{j}) \cdot
    \mathbf{u}(-{1\over 2}{1\over 2}0)\right]{1\over \sqrt{2}}(-\hat{i}
    + \hat{j})\\
  &=- {\alpha\over 2} \left[ (-\hat{i} + \hat{j}) \cdot
    \mathbf{u}(000) - (-\hat{i} + \hat{j}) \cdot
    \mathbf{u}(-{1\over 2}{1\over 2}0)\right](-\hat{i}
    + \hat{j})\\
  &=- {\alpha\over 2} \left[-u_x(000) + u_y(000) + u_x(000) - u_y(000) \right](-\hat{i}
    + \hat{j}) \label{eq:F2}\\
  &= 0
\end{align}


For $\mathbf{r}_n = a({1\over
  2}-{1\over 2}0)$:

\begin{equation}
  \mathbf{\hat{R}} = {1\over \sqrt{2}} (\hat{i} - \hat{j})
  \end{equation}

and  assuming
\begin{align}
  \mathbf{u}({1\over  2}-{1\over 2}0) &= \mathbf{A}e^{i \mathbf{k} \cdot \mathbf{r}_n - i \omega t}\\
  &= \mathbf{A}e^{i ({k a\over 2 \sqrt{2}} -{k a\over 2 \sqrt{2}} ) - i \omega t}\\
  &= \mathbf{A}e^{i 0 - i \omega t}\\
  &= \mathbf{u}(000) \\
  \end{align}

\begin{align}
  \mathbf{F} &= - \alpha \left[ \mathbf{\hat{R}}\cdot \mathbf{u}(000) -
  \mathbf{\hat{R}}\cdot
  \mathbf{u}(\mathbf{R})\right]\mathbf{\hat{R}}\\
  &=- \alpha \left[ {1\over \sqrt{2}} (\hat{i} - \hat{j}) \cdot
    \mathbf{u}(000) - {1\over \sqrt{2}} (\hat{i} - \hat{j}) \cdot
    \mathbf{u}({1\over 2}-{1\over 2}0)\right]{1\over \sqrt{2}}(\hat{i}
    - \hat{j})\\
  &=- {\alpha\over 2} \left[ (\hat{i} - \hat{j}) \cdot
    \mathbf{u}(000) - (\hat{i} - \hat{j}) \cdot
    \mathbf{u}({1\over 2}-{1\over 2}0)\right](\hat{i}
    - \hat{j})\\
  &=- {\alpha\over 2} \left[u_x(000) - u_y(000) - u_x(000) + u_y(000) \right](\hat{i}
    - \hat{j}) \\
&=0\\\end{align}

Makes sense since these atoms lie on the same wavefront and so should
always be in phase with the atom at the origin.

For $\mathbf{r}_n = a(-{1\over
  2},-{1\over 2},0)$:

\begin{equation}
  \mathbf{\hat{R}} = {1\over \sqrt{2}} (-\hat{i} - \hat{j})
  \end{equation}

and  assuming
\begin{align}
  \mathbf{u}(-{1\over  2},-{1\over 2},0) &= \mathbf{A}e^{i \mathbf{k} \cdot \mathbf{r}_n - i \omega t}\\
  &= \mathbf{A}e^{i ({-k a\over 2 \sqrt{2}} -{k a\over 2 \sqrt{2}} ) - i \omega t}\\
  &= \mathbf{A}e^{i {-k a\over \sqrt{2}} - i \omega t}\\
  &= \mathbf{u}(000) e^{i {-k a\over \sqrt{2}}} \\
  \end{align}

\begin{align}
  \mathbf{F} &= - \alpha \left[ \mathbf{\hat{R}}\cdot \mathbf{u}(000) -
  \mathbf{\hat{R}}\cdot
  \mathbf{u}(\mathbf{R})\right]\mathbf{\hat{R}}\\
  &=- \alpha \left[ {1\over \sqrt{2}} (-\hat{i} - \hat{j}) \cdot
    \mathbf{u}(000) - {1\over \sqrt{2}} (-\hat{i} - \hat{j}) \cdot
    \mathbf{u}(-{1\over 2},-{1\over 2},0)\right]{1\over \sqrt{2}}(-\hat{i}
    - \hat{j})\\
  &=- {\alpha\over 2} \left[ (-\hat{i} - \hat{j}) \cdot
    \mathbf{u}(000) - (-\hat{i} - \hat{j}) \cdot
    \mathbf{u}(-{1\over 2},-{1\over 2},0)\right](-\hat{i}
    - \hat{j})\\
  &=- {\alpha\over 2} \left[-u_x(000) - u_y(000) + u_x(000) e^{-i {k a\over \sqrt{2}}} + u_y(000) e^{-i {k a\over \sqrt{2}}} \right](-\hat{i}
    - \hat{j}) \\
    &=- {\alpha\over 2} \left[ u_x(000)\left( -1 + e^{-i ({k
    a\over \sqrt{2}} )} \right) + u_y(000)\left(-1+ e^{-i ({k a\over \sqrt{2}} )}\right) \right](-\hat{i}
    - \hat{j}) \label{eq:F4}\\
\end{align}


For $\mathbf{r}_n = a({1\over
  2},0,{1\over 2})$:

\begin{equation}
  \mathbf{\hat{R}} = {1\over \sqrt{2}} (\hat{i} + \hat{k})
  \end{equation}

and  assuming
\begin{align}
  \mathbf{u}({1\over  2},0,{1\over 2}) &= \mathbf{A}e^{i \mathbf{k} \cdot \mathbf{r}_n - i \omega t}\\
  &= \mathbf{A}e^{i ({k a\over 2 \sqrt{2}} + 0 ) - i \omega t}\\
  &= \mathbf{A}e^{i {k a\over 2 \sqrt{2}} - i \omega t}\\
  &= \mathbf{u}(000) e^{i {k a\over 2 \sqrt{2}}} \\
  \end{align}

\begin{align}
  \mathbf{F} &= - \alpha \left[ \mathbf{\hat{R}}\cdot \mathbf{u}(000) -
  \mathbf{\hat{R}}\cdot
  \mathbf{u}(\mathbf{R})\right]\mathbf{\hat{R}}\\
  &=- \alpha \left[ {1\over \sqrt{2}} (\hat{i} + \hat{k}) \cdot
    \mathbf{u}(000) - {1\over \sqrt{2}} (\hat{i} + \hat{k}) \cdot
    \mathbf{u}({1\over 2},0,{1\over 2})\right]{1\over \sqrt{2}}(\hat{i}
    + \hat{k})\\
  &=- {\alpha\over 2} \left[ (\hat{i} + \hat{k}) \cdot
    \mathbf{u}(000) - (\hat{i} + \hat{k}) \cdot
    \mathbf{u}({1\over 2},0,{1\over 2})\right](\hat{i}
    + \hat{k})\\
  &=- {\alpha\over 2} \left[u_x(000) + u_z(000) - u_x(000) e^{i {k a\over 2\sqrt{2}}} - u_z(000) e^{i {k a\over 2\sqrt{2}}} \right](\hat{i}
    + \hat{k}) \\
    &=- {\alpha\over 2} \left[ u_x(000)\left( 1 - e^{i ({k
    a\over 2\sqrt{2}} )} \right) + u_z(000)\left(1- e^{i ({k a\over 2\sqrt{2}} )}\right) \right](\hat{i}
    + \hat{k}) \label{eq:F4}\\
\end{align}

\end{document}